\documentclass{article}


\usepackage[utf8]{inputenc}
\usepackage[T1]{fontenc}
\usepackage[a4paper,includeheadfoot,margin=2.54cm,includehead]{geometry}
\usepackage[mono]{notomath}
\usepackage{float}
\usepackage{graphicx}
\usepackage{caption}
\usepackage{hyperref}
\usepackage{color}
\usepackage{fancyhdr}


\begin{document}

% ------------------------------------------------
% | % Título, autor, materia, fecha, ciudad... % |
% ------------------------------------------------
\newcommand{\titulo}{Informe}
\newcommand{\escuela}{Escuela de Ingeniería Informática}
\newcommand{\autorAngel}{Ángel Iglesias Préstamo}
\newcommand{\autorDiego}{Diego Martín Fernández}
\newcommand{\asignatura}{Informática Audiovisual}
\newcommand{\universidad}{Universidad de Oviedo}
\newcommand{\fecha}{\today} % Muestra por defecto la fecha actual

% ----------------------------------------------------------
% | % Establecemos un estilo para el encabezado y el pie % |
% ----------------------------------------------------------
\pagestyle{fancy}
\fancyhf{}

\renewcommand{\headrulewidth}{0.5pt}
\renewcommand{\footrulewidth}{0.5pt}

\fancyhead[LE]{\scshape\asignatura}
\fancyhead[RO]{\titulo}
\fancyfoot[CE,CO]{\thepage}

% -----------------------------
% | % Portada del documento % |
% -----------------------------
\begin{titlepage}
    \centering
    \includegraphics[width=0.6\textwidth]{img/uniovi.png}
    \vfill
    {\huge\bfseries\titulo\par}
    \vspace{3cm}
    {\Large\itshape\autorAngel\par\vspace{0.1cm}\autorDiego\par}
    \vspace{1.5cm}
    {\LARGE\escuela\par}
    \vspace{0.5cm}
    {\Large\asignatura\par}
    \vfill
    \universidad\par
    \fecha
\end{titlepage}

% -----------------------------------------
% | % Tabla de contenidos del documento % |
% -----------------------------------------
\renewcommand{\contentsname}{Tabla de contenidos}
\tableofcontents{}

% ---------------------------------------
% | % Empezamos el documento en sí :D % |
% ---------------------------------------


\end{document}
